\section{Lessons Learned}\label{section5}

During the execution of the activities documented in this paper, the authors identified some situations in which works related to the concepts involved can benefit. As lessons learned we highlight:

\begin{itemize}
    \item \textbf{Domain Characteristics:} a domain analysis can be considered one of the most important activities for the creation of an SPL. The evolution of the catalog of requirements proposed by Duarte Filho and Barbosa \cite{filho13} significantly contributed in terms of domain knowledge and supported the adoption of the proactive model for the development of M-SPLear\allowbreak ning.
    \item \textbf{Variabilities and M-Learning:} the use of the SMarty notation helps the identification of the variation points during the design of M-SPLear\allowbreak ning, ensuring greater cohesion for the implementation of the SPL components. It also contributes to the assimilation of the SPL concept. However, those involved must be trained so that the elements of notation can be used consistently.
    \item \textbf{M-SPLear\allowbreak ning Development:} the implementation of something generic and customizable is significantly different from a static approach. Therefore, developing features in an SPL requires a greater effort, which is justified by the subsequent gains of reuse.
    \item \textbf{M-SPLear\allowbreak ning Experimental Evaluation:} researches show that test executions in SPLs are scarce and need to be evaluated and validated \cite{engstrom11}. Thus, the authors decided to apply the test cases in the products generated by the SPL, enabling a interesting comparison with an alternative methodology of development.
  
    The experimental evaluation provided relevant results for the adoption of M-SPLear\allowbreak ning. The choice for active participants in the industry contributed to the reduction of the training session. However, experience and understanding the concepts by the participants is always difficult to measure.
\end{itemize}
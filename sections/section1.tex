\section{Introduction}

Constant changes in software reuse approaches have lead to the concept of Software Product Line (SPL), which represents a shift in focus from the singular software development paradigm. Companies that had developed project-by-project software have now focused on the creation and maintenance of SPL and their variabilities. Therefore, models that represent variabilities are specified as part of the core assets of an SPL and their correct identification, specification and representation provide several development benefits \cite{chen11,capilla13}.

The variabilities management that enable diversification in the portfolio of products in a given domain requires the adoption of a well-defined systematic approach. Some of these approaches can be used in Domain Engineering (DE) and Application Engineering (AE) to support the selection and delimitation of the variant artifacts from different products \cite{bockle05,vanderlinden07}.

For an SPL to be successful, its domain must be carefully defined. If the domain is too large and product members vary widely, the core assets will be strained beyond their ability to accommodate the variation, economies of the production will be lost, and the product line will collapse into an old-style, one-at-a-time product development effort. On the other hand, if the domain is too small, the core assets might not be built in a generic enough fashion to accommodate a future growth and the product line will stagnate, i.e., economies of the domain will never be achieved and the full potential return of investment (ROI) will never materialize \cite{bockle05,vanderlinden07}.

From a different, but related perspective, the rapid growth of information and communication technologies has favored the emergence of innovative ways for facing the shortcomings of traditional education \cite{west12}. Mobile learning (m-learning) for instance, has provided a strong interaction between learners and instructors, enabling them to actively participate in the knowledge construction process anytime and anywhere \cite{kukulska05}. 

M-learning has grown in terms of importance and visibility, mainly because of significant results regarding flexibility and propagation of education \cite{kinshuk03,wexler08}. Such aspects have made m-learning a promising tool for education. In 2011, the first ``UNESCO Mobile Learning Week'' suggested m-learning as an alternative to the ``teacher crisis'', an expression justified by global need for 8.2 million new teachers for the achievement of the UN Millennium Development Goal of providing universal primary education by 2015 \cite{west12}.

Due to the diversity of platforms, technologies and pedagogical methods considered for the development of m-learning applications, a wide range of specificities can be streamlined and addressed from a reuse perspective. However, few studies have focused on development issues through a strategy of systematic reuse, such as SPL, for mobile middleware, hence the m-learning domain \cite{bezerra09}.

Motivated by this scenario, we have worked on the establishment of M-SPLear\allowbreak ning, an SPL to the m-learning domain \cite{falvojr14a,falvojr14b}. M-SPLear\allowbreak ning has been developed based on a concise UML-based variability management approach, named SMarty, which provides mechanisms to facilitate the identification and representation of variabilities \cite{oliveirajr10}.

This paper discusses how variability improves the development of m-learning applications by SMarty. We experimentally evaluated M-SPLear\allowbreak ning regarding singular software development, particularly comparing time-to-market and quality of the software products implemented to the use of both (SPL and singular) approaches. The results showed a significant reduction in the time-to-market and improvement in the quality in terms of faults, when considering the software products developed from the M-SPLear\allowbreak ning core assets with the support of variabilities.

The paper is organised as follows: Section \ref{section2} summarizes the background; Section \ref{section3} describes M-SPLear\allowbreak ning; Section \ref{section4} addresses the experimental evaluation; Section \ref{section5} discusses the lessons learned from the development and application of M-SPLear\allowbreak ning; Section \ref{section6} reviews the related works; finally, Section \ref{section7} provides the conclusions and perspectives for future work.

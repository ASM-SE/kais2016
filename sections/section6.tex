\section{Related Work} \label{section6}

%There is a lack on literature with regard to the use of variabilities and SPL to create mobile learning applications, as it is possible to note by searching terms as ``mobile learning software product line'', ``mobile learning variabiliti'' and variations, in ACM\footnote{http://dl.acm.org/} or IEEE\footnote{http://ieeexplore.ieee.org/} search bases. Therefore, the related works has a different focus of mobile learning, although, they specify and present some open issues to be researched and explored.

Gamez et al. \cite{gamez14} proposed a self-adaptation of mobile systems with dynamic SPLs. The management of variabilities is achieved using the Common Variability Language (CVL). The authors claim that CVL allows modeling the variability separately from the base model, but both the variability and the base models appear as connected and can be managed using the same tool. %The proposal was applied in a case study and presents good results.

Marinho et al. \cite{marinho10} proposed an architecture for nested SPLs in the domain of mobile and context-aware applications. However, the authors did not specify how to improve the management of variabilities. %The strategy used was defined a nested SPL that aims to facilitate the construction of such software by domain decompostition in two level of analysis. 

Bezerra et al. \cite{bezerra09} conducted a systematic review of SPLs applied to mobile middlewares, but only six studies were significant for the review. According to the authors, the few results obtained highlight the need of more research in the area. %In the specification of techniques used to developed SPLs in to mobile middleware, standed out feature model and domain specific language.

In another systematic literature review, Chen and Babar \cite{chen11} concluded the status of evaluation of variability management approaches in SPL engineering was quite dissatisfactory.

Actually, there are several opportunities and open issues regarding the management of variabilities and experimental evaluation, particularly in the m-learning domain. Such needs of research in the area has motivated our work.

%It is clearly visible that the related works is not directly related with our study. This few works motivated the research to analise the variabilities in the mobile learning context. The paucity of researchs presents the necessity to investigate the mainly issues and opened oportunities to explore them, such as the application of them in educational context. Other visible lack in the literature is related the experimental validation to compare the time-to-market and quality of products developed with SPL and SSD methodologies, as metioned by Chen and Muhammad \cite{chen2011systematic}.

%As can be seen, there is a need for investigating variability management in several context, research  to investigate the mainly issues and opened oportunities to explore them, such as the application of them in educational context lack of As pointed out by Chen and Muhammad \cite{chen2011systematic}. Other visible lack in the literature is related the experimental validation to compare the time-to-market and quality of products developed with SPL and SSD methodologies, as metioned by Chen and Muhammad \cite{chen2011systematic}.

%The needs to explore them and experimentally evaluation the real beneficts of this methodology it is justify, not only by the reports in software product line studies in other domains, but the possibilite of the adoption of a variability management approach and SPL paradigm as a new oportunity to replace the traditional single development.

%A incipiência de estudos experimentais no contexto de linhas de produto, que comparem este paradigma de reutilização de artefatos de software com outras metodologias de desenvolvimento é claramente visível, como menciona Chen e Muhamad em sua revisão sistemática da literatura \cite{chen2011systematic}. A escassez de trabalhos neste sentido levou a busca de trabalhos isolados de avaliação de paradigmas de desenvolvimento para a condução do presente estudo. 

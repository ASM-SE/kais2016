\section{Related Work} \label{section6}

Our work encompass three main perspectives in industry environments: (i) m-learning applications, (ii) SPL with its benefits through variability management and (iii) Experimental Software Engineering. Adopting SPL as methodology to develop m-learning applications, allowed us to collect positive evidences in a real industry environment about two measurable SPL benefits: quality of products and time-to-market, when compared experimentally with a singular software development process without a variability management approach to support the developers. Thus, our results come to complement the experiences related in other researches for these three perspectives.

In mobile applications development domain, Gamez et al. \cite{gamez14} proposed a self-adaptation of mobile systems with dynamic SPL. The management of variabilities is achieved using the Common Variability Language (CVL). Marinho et al. \cite{marinho10} proposed an architecture for nested SPL in the domain of mobile and context-aware applications. However, the authors did not specify how to improve the management of variabilities. Bezerra et al. \cite{bezerra09} conducted a systematic review of SPL applied to mobile middlewares, but only six studies were significant for the review. According to the authors, the few results obtained highlight the need of more research in the area. 

These evidences led us to align the necessity to grow the number of m-learning applications -- which presents a still incipient exploration in literature -- using a reused-based approach. In our case, SPL methodology with SMarty to manage the variabilities. Analysing the literature, as for m-learning development, there is also a lack in researches about the use of SPL and variability management as a way to improve quality and time-to-market in industry. Chen and Babar \cite{chen11}, in a systematic literature review, concluded the status of evaluation of variability management approaches in SPL engineering was quite unsatisfactory. 

Jaring and Bosch \cite{jaring02} presents a case study in a Dutch-based company, Rohill Technologies BV -- a system development for, mainly, professional mobile communication infrastructures. They discussed and analyzed the need for handling variability in a more explicit manner, discussing the SPL and a methods to represent and normalize variabilities. Some issues were highlighted, as limited insight into the consequences of selecting a particular variability mechanism for variability identification; a notation format to describe variability is not available and dependencies between variability and features are not made explicit to variability dependencies.

Hubaux et al \cite{hubaux10} combined variability representation and industrial case studies evaluations. They developed a Textual Variability Language (TVL) combining graphical and textual notations and conducted an evaluation through a quantitative and qualitative analysis in four cases from different companies, sizes and domains. Positive benefits for TVL were identified, as efficiency gained in terms of model comprehension, design and learning curve for the notation. Some limitations were presented as well. Eriksson et al \cite{eriksson09} also presented a language, but to manage requirements specifications for SPL. Using a qualitative evaluation -- document examination, participant observation and semi-structure interviews -- they compared the clone-an-own reuse with the proposed approach, which must be use together with a previously developed method to manage product line use cases models. Six variables were analyzed: adoption effort, expressiveness, scalability, ROI, risks and reuse infrastructure. The ROI variable presented negative results, the others presented positive evidences for the proposed language. 

Still in the context of SPL evaluation in industry, but with focus on quality through software testing, Ardis et al \cite{ardis00} used the Family-Oriented Abstraction, Specification and Translation (FAST) approach as a development process for an SPL in a case study, covering all aspects of domain analysis with tests. According to the authors, test process in an SPL presents significant challenges. Thus, they presented, for each case study, three test strategies suitable for general use in SPL testing: (i) testing common code thoroughly; (ii) exploiting common aspects of variable code; and (iii) utilize scenarios from the commonality analysis. 

Gacek et al \cite{gacek01} presents a study case in the adoption of SPL in a small company called Market Maker. An holistic view of the challenges and changes in business was discussed, especially the automatization of tests in their developed components. Firstly, each single component was tested and, secondly, the entire system at runtime, using a special code inserted in each component. If an error occurred, the inserted test code identifies automatically which component misbehave. The testing process was conducted only for components that are present in the instantiated product, in other words, components are not tested in their full flexibility, with respect to all possible instantiations in the family.

Batory et al \cite{batory02} define a SPL architecture with a Domain-Specific Language (DSL) to redesign an extensible command-and-control simulator for army fire support. In the case study, they collected preliminary results that Product Line Architecture (PLA) with DSL produce a more flexible way to implements the simulator and reduced the program complexity,if compared with the same simulator, with pure Java implementation. The complexity of the code was compared based in the number of methods, number of line codes and tokens/symbols for both the adopted approach and the Java implementation.

Finally, observing the related works it was noted that, in spite of all of them were applied in industry, only few of them compare methodologies, as SPL and singular development, highlighting which of them bring more quality and reduces time-to-market, when supported by a variability management approach. Additionally, there are several opportunities and open issues regarding the management of variabilities and experimental evaluation, particularly in the m-learning domain. Such needs of research in the area has motivated our work.